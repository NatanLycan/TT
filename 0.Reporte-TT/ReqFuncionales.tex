%%%%% Requerimientos Funcionales %%%%%
\subsection{Requerimientos funcionales}

\subsubsection{Requerimientos funcionales del usuario}
\begin{itemize}
%% sugerida, facil, solo es una conversion, se eligiria la unidad antes de tomar la foto
\item \textbf{RF\_U1} El usuario podrá seleccionar las unidades (cm., mts.,inch., etc) en las que quiere ver las dimensiones.
%% sugerida, se va a mostrar los resultados de todos modos, se puede guardar la foto con las medidas
\item \textbf{RF\_U2} El usuario podrá guardar las dimensiones obtenidas.
%%creo que este es más un no funcional, de hecho la agregare a requerimientos no funcionales  --> El usuario podrá establecer un codigo QR de referencia para calcular las medidas.
%%\item \textbf{RF\_U2} El usuario podrá modificar un plano virtual para establecer los bordes del objeto.
\item \textbf{RF\_U3} El usuario podra acceder a un tutorial acerca de como usar la aplicación.
\item \textbf{RF\_U4} El usuario podra ver información acerca de la aplicación y los desarrolladores.
\end{itemize}

\subsubsection{Requerimientos funcionales del sistema}
\begin{itemize}
%% sugerida, facil, solo es una conversion, se eligiria la unidad antes de tomar la foto
\item \textbf{RF\_S1} La aplicación solicitara que se realice una calibracion al abrirla.
\item \textbf{RF\_S2} La aplicación solicitara las medidas del codigo QR de referencia.
\end{itemize}